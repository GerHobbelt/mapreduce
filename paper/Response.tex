\documentclass[11pt]{article}
\usepackage[pdftex]{graphicx}
\usepackage{url}

\setlength{\textwidth}{6.5in}
\setlength{\textheight}{9.25in}

\setlength{\voffset}{-0.0in}
\hoffset=-50pt
\topmargin=0pt
\headheight=0pt
\headsep=0pt
\evensidemargin=72pt % 1in+7pt

\setlength{\parskip}{0.5ex}
\setlength{\parindent}{1em}
\setlength{\floatsep}{1pt}
\setlength{\textfloatsep}{2pt}
\setlength{\intextsep}{2pt}

\begin{document}

October 29, 2010\\

Response to reviewers:  \\

We thank the reviewers for their careful reading of our paper.  Their
suggestions definitely strengthen the paper and sparked new ideas for
improving our algorithms.  Below we list the Reviewers' comments
verbatim.  Our responses are shown in italics. \\

Steve Plimpton and Karen Devine

\vspace{0.5 in}

Reviewer \#1: The submitted paper features a description and analysis of a
new MapReduce implementation based on MPI.  The features and limitations
of the implementation are covered with respect to other MapReduce
implementations.  The implementation is applied to several
graph-theoretic algorithms, and the performance of these applications
is compared with applications from an established library (Trilinos).

The introductory and supporting text are, overall, very good.  The
paper places MapReduce in its historical context and motivates its
usage.  The outline of strengths is helpful- important features
include support for out-of-core operations that other libraries (e.g.,
Trilinos) often cannot handle and the ability to use MPI directly in
addition to the MapReduce abstraction.  The paper then provides a good
overview of the composition of operations in a MapReduce application
and a summary of the out-of-core mechanisms available.  The figure
(Figure 1) used to provide a high-level schematic of MapReduce data
flows could be enhanced- for example, it should indicate node-local
and communication intensive operations.

{\it Figure 1 is actually describing an on-processor rearrangement of
data that each processor performs on its local data it owns.  Thus
there is no inter-processor communication involved.}

All performance results are shown as scaling studies for a range of
problem sizes for the six graph algorithms.  The results show that the
MapReduce applications do not compete well with the reference
implementation (Trilinos), but the authors make clear the limitations
involved in using this high-level library which trades some
performance for usability.  Additionally, the implementation supports
very large data sets which can only be processed using its out-of-core
feature set.  The results are presented well.

Since improved performance for small and mid-range problems is not the
contribution of the paper, it could be improved by stressing other
aspects of the system.  Real-world out-of-core problems could be
surveyed.  Code size and complexity could be considered with usability
as a motivation.  Is the MapReduce-based graph algorithm library
easier to use than the Trilinos library?  Should higher-level user
applications be considered?  Can the MapReduce-based inefficiencies be
isolated more concisely?

The paper does not have a concise "Conclusion", only a "Lessons Learned"
section.

{\it We have made some of these comparison points more
clearly in the final ``Lessons Learned'' section.}

%comment on MR vs Trilinos, even if libraries already exist
%  cannot use linear algebra lib for all graph algorithms described
%  here - PR is an exception since can be cast as matvec
%added line counts of algorithms

\vspace{0.5 in}

Reviewer \#2: This is a nice paper addressing an important current topic
with a good quality open-source implementation.  The overall presentation
is good: the description of algorithms and implementations is clear and the
performance evaluation is detailed. As the authors pointed out, the
performance of the MR-MPI implementations of some graph algorithms falls
behind the matrix-based counterparts so it would be interesting to further
investigate where the performance descrepency comes from and if it's
possible to optimize the algorithm and implementation to bridge the
performance gap.

{\it We instrumented our MR-MPI and Trilinos implementations of
PageRank to show where time is spent during the PageRank iterations.
These results were added in Table 2.  We further instrumented the two
most expensive steps of the PageRank iterations, as shown in Tables 3
and 4.  These experiments showed that considerable time is spent in
reorganizing data from key-value pairs to key-multivalue pairs.  In
sections 4.1 and 5.2, we describe alternative data layouts that could
reduce the number of times the data must be reorganized to improve
performance.}

\vspace{0.5 in}

Reviewer \#3: I appreciated the clear exposition of your map/reduce
framework, its decomposition into a number of independent and composable
stages, and the explanation of how out-of-core computations are handled.

I also liked that you presented the map/reduce implementation of a
number of different graph algorithms. However, I thought the
presentation of these algorithms could be improved in two ways: - The
text tends to just explaining what the code does, not why, nor why
this solves the specific problem (4.2 was the worst culprit here - f I
understood correctly, the vertices get labeled by edge counts that are
used to give the edges a direction that helps identify triangles: if
this is correct a little diagram showing this would help...)  - The
detailed algorithms (Figures 4-9) are a little too imprecise to follow
at times. In particular, when multiple map/reduces are involved the
lack of names for the map/reduce operations and their inputs makes
things confusing. Please be very explicit (after all the code itself
is), and Figure 4 at least actually had names for the map-reduce
objects.

{\it We significantly re-worked and standardized the notation used
in Section 4 to describe all the graph algorithms.  In particular
we noted which operations are performed with which MapReduce
objects and their associated key/value data, so that the flow
of data through each algorithm should be more clear.  A discussion
of the MR-MPI library's ability to allow creation of multiple
MapReduce objects was added to Section 2.}

Your performance is quite low when compared to non map/reduce
implementations of the same algorithm. It would be interesting to know the
breakdown of execution time between map, shuffle and reduce phases to get a
better understanding of where your system spends its time (not necessarily
for every single performance measurement, but some representative or
overall measurements would be interesting).

{\it Reviewer \#2 made this same suggestion; please see the response above.}

Minor:
- 5.4: In Figure *13* (not 6)...

{\it Fixed this; good catch.}

\end{document}
