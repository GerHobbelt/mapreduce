\section{Introduction}
\label{sec:intro}

MapReduce is the programming paradigm popularized by Google
researchers Dean and Ghemawat~\cite{MapReduce2004}.  Their motivation was to
enable rapid development and deployment of analysis programs to
operate on massive data sets residing on Google's large distributed
clusters.  They introduced a novel way of thinking about certain kinds
of large-scale computations as a ``map'' operation followed by a
``reduce'' operation.  The power of the paradigm is that when cast in
this way, a nominally serial algorithm now becomes two highly parallel
operations working on data local to each processor, sandwiched around
an intermediate data-shuffling operation that requires inter-processor
communication.  The user need only write serial code for the
application-specific map and reduce functions; the parallel data
shuffle can be encapsulated in a library since its operation is
independent of the application.

The Google implementation of MapReduce is a C++ library with
communication between networked machines via remote procedure calls.
It allows for fault tolerance when large numbers of machines are used,
and can use disks as out-of-core memory to process petabyte-scale data
sets.  Tens of thousands of MapReduce programs have since been written
by Google researchers and are a significant part of the daily compute
tasks run by the company~\cite{MapReduce2008}.

Similarly, the open-source Hadoop implementation of MapReduce~\cite{Hadoop}, 
has become widely popular in the past few years for
parallel analysis of large-scale data sets at Yahoo and other
data-centric companies, as well as in university and laboratory
research groups, due to its free availability.  MapReduce programs in
Hadoop are typically written in Java, although it also supports use of
stand-alone map and reduce kernels, which can be written as shell
scripts or in other languages.

More recently, MapReduce formulations of traditional number-crunching
kinds of scientific computational tasks have been described, such as
post-processing analysis of simulation data~\cite{Tu08}, graph
algorithmics~\cite{Cohen09}, and linear algebra operations~\cite{Ekanayake09}.
The paper by Tu et al.~\cite{Tu08} was particularly insightful to us,
because it described how MapReduce could be implemented on top of the
ubiquitous distributed-memory message-passing interface (MPI), and how
the intermediate data-shuffle operation is conceptually identical to
the familiar MPI\_Alltoall operation.  Their implementation of
MapReduce was within a Python wrapper to simplify the writing of user
programs.  The paper motivated us to develop our own C++ library built
on top of MPI for use in graph analytics, which we initially released
as open-source software in mid-2009~\cite{MRMPI}.  We have since
worked to optimize several of the library's underlying algorithms and
to enable its operation in out-of-core mode on larger data sets.
These algorithmic improvements are described in this paper and are
part of the current downloadable version~\cite{MRMPI}.

The MapReduce-MPI (MR-MPI) library described in this paper is a
simple, lightweight implementation of basic MapReduce functionality,
with the following features and limitations:

\begin{itemize}

\item {\it C++ library using MPI for inter-processor communication:}
The user writes a (typically) simple main program which runs on each
processor of a parallel machine, making calls to the MR-MPI library.
For map and reduce operations, the library calls back to user-provided
{\it map()} and {\it reduce()} functions.  The use of C++ allows
precise control over the memory and format of data allocated by each
processor during a MapReduce.  Library calls for performing a map,
reduce, or data shuffle, are synchronous, meaning all the processors
participate and finish the operation before proceeding.  Similarly,
the use of MPI within the library is the traditional mode of MPI\_Send
and MPI\_Recv calls between pairs of processors using large aggregated
messages to improve bandwidth performance and reduce latency costs.  A
recent paper~\cite{Hoefler09} also outlines the MapReduce formalism
from an MPI perspective, although they advocate a more asynchronous
approach, using one-way communication of small messages.

\item {\it Small, portable:} The entire MR-MPI library is a few
thousand lines of standard C++ code.  For parallel operation, the
program is linked with MPI, a standard message passing library
available on all distributed-memory machines and many shared-memory
parallel machines.  For serial operation, a dummy MPI library
(provided) can be substituted.  As a library, MR-MPI can be embedded
in other codes~\cite{Titan} to enable them to perform MapReduce
operations.

\item {\it In-core or out-of-core operation:} Each MapReduce object
that a processor creates allocates ``pages'' of memory, where the page
size is determined by the user.  Typical MapReduce operations can be
performed using a few such pages (per processor).  If the data set
fits in a single page (per processor), the library performs its
operations in-core.  If the data set exceeds the page size, processors
each write to temporary disk files (on local disks or a parallel file
system) as needed and subsequently read from them.  This out-of-core
mode of operation allows processing of data sets larger than the
aggregate memory of all the processors, i.e. up to the aggregate
available disk space.

\item {\it Flexible programmability:} An advantage of writing a
MapReduce program on top of MPI, is that the user program can also
invoke MPI calls directly, if desired.  For example, one-line calls to
MPI\_Allreduce are often useful in determining the status of an
iterative graph algorithm, as described in Section~\ref{sec:graph}.
The library interface also provides a user data pointer as an argument
passed to all callback functions, so it is easy for the user program
to store ``state'' on each processor, accessible during the map and
reduce operations.  For example, various flags can be stored that
alter the operation of a map or reduce operation, as can richer data
structures that accumulate results.

\item {\it C++, C, and Python interfaces:} The C++ interface to the
MR-MPI library allows user programs to instantiate and then invoke
methods in one or more MapReduce objects.  The C interface allows the
same operations to be called from C or other high-level languages such
as Fortran.  The C interface also allows the library to be easily
wrapped by Python via the Python ``ctypes'' module.  The library can
then be called from a Python script, allowing the user to write {\it
map()} and {\it reduce()} callback functions in Python.  If a machine
supports running Python in parallel, a parallel MapReduce can also be
run in this mode.

\item {\it No fault tolerance or data redundancy:} Current MPI
implementations do not enable easy detection of a dead processor or
retrieval of the data it was working on.  So like most MPI programs, a
parallel program calling the MR-MPI library will hang or crash if a
processor goes away.  Unlike Hadoop, and its HDFS file system which
provides data redundancy, the MR-MPI library reads and writes simple,
flat files.  It can use local per-processor disks, or a parallel file
system, if available, but these typically provide no data redundancy.

\end{itemize}

The remainder of the paper is organized as follows.
Sections~\ref{sec:mr} and~\ref{sec:outcore} describe how in-core and
out-of-core MapReduce primitives are formulated as MPI-based
operations in the MR-MPI library.  Section~\ref{sec:graph} briefly
describes the formulation of several common graph algorithms as
MapReduce operations.  Section~\ref{sec:results} gives performance
results for these algorithms running on a parallel cluster for graphs
ranging in size from 8 million to 2 billion edges; we highlight the
performance and complexity trade-offs of a MapReduce approach versus
other more special-purpose algorithms.  The latter generally perform
better but are harder to implement efficiently on distributed memory
machines, due to the required explicit management of parallelism,
particularly for large out-of-core data sets.
Section~\ref{sec:lessons} summarizes some lessons learned from the
implementation and use of our library.
